\documentclass{article}
\usepackage[utf8]{inputenc}
\usepackage{comment}

\title{Motion smoothing for Skylar}
\author{Ian Huang}
\date{\today}

\begin{document}
\maketitle
\section{Introduction}
One of the earlier mistakes made was not accounting for the variable amounts of
current draw in the circuits when falsely parsed commands were given to the
arduino. On one occasion, this actually caused one all of the servos to fail,
and possibly one of the PWM pins on the arduino. To remedy this, I plan to
smooth the motion of the robot.

The smoothing will be parabolic, and will have the form:\\
$$\omega(t)= at^2 + bt + c$$
From this, given $\tau$, the time period in which we'd like all of the joints
to transition to their next state ($\theta_1$ and $\theta_1'$ stand for the 
prior and post-transition joint angles of the first joint), we would like to
find a formula for $theta_i(t)$ for every joint $i$.

\section{Derivations}
% some more math
We know that \\
1. $\int_{0}^{\tau}\omega_i(t) dt = \theta_i' - \theta_i$\\
2. $\omega_i(0) = 0$ \\
3. $\omega_i(\tau) = 0$\\

This allows us to solve for $a_i$, $b_i$, and $c_i$ in terms of $\tau$, 
$\theta_i$, and $\theta_i'$. It becomes apparent that $c = 0$ from the 2nd
condition.

Expanding the integral in the first condition, we get:
$$\biggr[ \frac{1}{3}a_it^3 + \frac{1}{2}b_it^2 \biggr|^\tau_0 = \Delta \theta_i$$
Plugging in $\tau$ to the third condition, we get:
$$a\tau^2 + b\tau = 0$$
$$b = -a\tau$$
Plugging this into the expanded first condition:
$$\frac{1}{3}a\tau^3 + \frac{1}{2}(-a\tau)\tau^2 = \Delta \theta_i$$
$$a = \frac{-6(\theta_i' - \theta_i)}{\tau^3}$$
Since we know that:
$$b=-a\tau$$
we can solve for b:
$$b = \frac{6(\theta_i' - \theta_i)}{\tau^2}$$
To solve for $\theta_i(t)$, we have:
$$\theta_i(t) = \int \biggr( \frac{-6(\theta_i' - \theta_i)}{\tau^3} t^2 + \frac{-6(\theta_i' - \theta_i)}{\tau^3} t \biggr) dt $$
$$\theta_i(t) = \frac{-2(\theta_i'-\theta_i)}{\tau^3} t^3 + \frac{3(\theta_i' - \theta_i)}{\tau^2} t^2 + C$$
We solve for the integration constant at $t = 0$,
$$\theta_i(0) = C = \theta_i $$
Therefore, for prior angle vector $\Theta$, destination angle vector
$\Theta'$ and time duration $\tau$, we can calculate the coefficients in the
equation for $\theta(t) = \alpha t^3 + \beta t^2 + C$, in vectorized form. \\
$$\alpha = \frac{-2}{\tau^3}(\Theta' - \Theta)$$
$$\beta = \frac{3}{\tau^2}(\Theta' - \Theta)$$
$$C = \Theta$$

\section{End Result}
% Photos of end result? How well does it work

\end{document}
